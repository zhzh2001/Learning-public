\documentclass[hyperref,UTF8,12pt,a4paper]{ctexart}

\usepackage{amsmath,algorithm}

\usepackage{geometry}
\geometry{left=1in,right=1in,top=1in,bottom=1in}

\usepackage{titling}
\pretitle{\begin{center}\fontsize{30pt}{30pt}\selectfont}
\posttitle{\end{center}}

\usepackage{algpseudocode}
\floatname{algorithm}{算法}

\title{NOIp Senior Day1 Solution}
\author{zhzh2001}
\date{}

\begin{document}

\maketitle

\section{阶乘}

\subsection{直接计算}

用64位整数、双精度浮点数或扩展精度浮点数保存结果,时间复杂度$O(n)$。

期望得分:5/15/30

\subsection{高精度}

直接用高精度计算,时间复杂度大约$O(n^2)$,用压位可以加速。

期望得分:50(使用huge模板)\textasciitilde{}65(使用gmp库)

\subsection{计算对数}

使用\verb|log10|函数即可计算出$n!$以10为底的对数,取整作为指数,小数部分作为尾数(用\verb|pow|计算),时间复杂度$O(n)$。

然而即使用扩展精度浮点数也有误差。

期望得分:71

\subsection{正解}

其实也很简单,由于只要保留$k$位有效数字,后面很多数字显然都没有用。考虑仍然用浮点数保存结果,但是这次,每次算完后都判断结果是否大于一个特定值($10^k$),如果是则除以这个值,并累加计数器。经过实验,这种方法基本上是正确的(用高精度浮点数库\verb|mpfr|对拍)。时间复杂度$O(n)$。

然而,用双精度浮点数保存结果精度不够,只能得到60分。应该用扩展精度浮点数保存才能得到95分。最后一个点需要分段打表。

\subsection{使用lgamma函数计算}

\verb|gamma|函数定义为$\Gamma n=(n-1)!$,有一个函数\verb|lgamma|可以计算\verb|gamma|函数的自然对数,换底之后就是答案了。但是即使用扩展精度浮点数精度也不够,要用四精度浮点数,在GCC中为\verb|__float128|。但是四精度的数学运算需要特殊的\verb|quadmath.h|数学库,而且在连接时要加上\verb|-lquadmath|选项。唯一的方法是把源代码复制到代码中,但是这样代码就超过长度限制了,需要进行精简。用这种方法的标程,有大约18KB长。时间复杂度是$O(1)$的。

\subsection{总结}

这题是我的原创题,方法也比较多。但是数据范围可能比较神奇。

主要考察了创新能力和分段打表。

\section{激光和镜子}

\subsection{思路}

在这个问题中,我们想要把激光从源点发射到终点。激光可以从水平或竖直方向开始,并且镜子可以放置在特定位置来改变激光的方向。我们想要放置最少的镜子来到达终点。

我们可以发现在任何激光的最优路径中,任意一条水平或竖直的直线,激光最多只会覆盖直线上连续的一段。如果激光覆盖了分离的两段,那么我们可以跳过中间转弯的部分,并且找到一种更优的路径。

因此,激光最多只能在$2N+2$条直线上行进————$N+1$条水平直线和$N+1$条竖直直线,与源点和$N$个可以放镜子的点对应。

于是我们可以把这个问题转化为最短路问题。我们想要到达一条经过终点的水平或竖直的直线,并且我们从一条经过源点的水平或竖直的直线出发。我们可以在两条直线间切换,当且仅当这两条直线的交点是给定的$N$个点之一,并且我们想要最小化切换的次数。

\subsection{技巧}

由于坐标范围很大,我们需要把坐标离散化后再处理。

另外,由于所有的边权都为1,可以用\verb|bfs|来实现最短路,而不用\verb|Dijkstra|,而且时间复杂度更优,最短路部分为$O(N)$。排序的常数比\verb|Dijkstra|小多了。

\subsection{总结}

本题来自\verb|USACO16DEC Gold T3:Lasers and Mirrors|

主要考察了图的建立以及最短路,其中把点转为边的思想比较巧妙。

\section{干草堆猜测}

\subsection{思路}

我们可以发现,如果有一种方案满足问题$1\dots M$,那么同一种方案也满足问题$1\dots M-1$。所以我们可以二分答案,并把问题转化为确定一组问题是否能满足的判定性问题。

\subsection{所有的A互不相同}

考虑出现矛盾的充要条件,设当前问题为$Q_l,Q_r,A$,如果$Q_l\dots Q_r$间的所有草堆的答案都大于$A$,那么存在矛盾,因为$A$为RMQ($Q_l,Q_r$)的上界。而如果满足条件,那么其他位置都可以放置任意值而不造成任何问题。把问题按照$A$降序排序,对于每个问题先判断再染色,用数据结构维护区间染色即可。

\subsection{所有的A相同}

只要所有问题的区间有交就可以满足条件,反之显然草堆的数量重复出现,不符合题意。

\subsection{所有情况}

对于$A$相同的问题一起处理,先判断区间交是否被完全染色,再染色区间并。用并查集实现区间染色(当然线段树也可以),一次判定的时间复杂度为$O(N+Q\log Q)$。实际上只要开始时排序,判定时取出符合条件的问题即可,总时间复杂度$O(N\log Q)$。

官方的方法有些不同,时间复杂度是$O(N\log Q+Q\log^2 Q)$,无法通过数据加强。

\subsection{并查集维护区间染色}

维护$N+1$个点的并查集$0\dots N$,每个点的父亲表示从这个点出发染色段的左端,初始时$f[i]=i$。利用这些信息很容易实现查询。

\begin{algorithm}
\caption{}
\begin{algorithmic}[1]
\Procedure{Paint}{$f,L,R$}\Comment{$f$为并查集,$[L,R]$为染色区间}
\While{\Call{Root}{$L-1$}$\not=$\Call{Root}{$R$}}
\State{$f$[\Call{Root}{$R$}]$\gets$\Call{Root}{$L-1$}}
\EndWhile
\EndProcedure
\end{algorithmic}
\end{algorithm}

算法1可能因为递归的\verb|Root|而栈溢出,为了避免这个问题,还有另一种算法2。

\begin{algorithm}
\caption{}
\begin{algorithmic}[1]
\Procedure{Paint}{$f,L,R$}
\While{$L\le R$}
\If{\Call{Root}{$R$}=$R$}
\State{f[$R$]$\gets$\Call{Root}{$L-1$}}
\State{$R\gets R-1$}
\Else
\State{$R\gets$\Call{Root}{$R$}}
\EndIf
\EndWhile
\EndProcedure
\end{algorithmic}
\end{algorithm}

很明显,这里不能用按秩合并,只能用路径压缩。因此理论时间复杂度与线段树相同,但是常数小,且实现简单。可以做\verb|codevs1191|来练习。

\subsection{总结}

本题来自\verb|USACO08JAN Gold T1:Haybale Guessing|

\end{document}
