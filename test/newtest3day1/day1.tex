\documentclass[hyperref,UTF8,12pt,a4paper]{ctexart}

\usepackage{amsmath,makecell,multirow}

\usepackage{array}
\newcolumntype{P}[1]{>{\centering\arraybackslash}p{#1}}

\usepackage{geometry}
\geometry{left=1in,right=1in,top=1in,bottom=1in}

\usepackage{titling}
\pretitle{\begin{center}\fontsize{30pt}{30pt}\selectfont}
\posttitle{\end{center}}

\title{NOIp Senior Day1}
\author{zhzh2001}
\date{}

\begin{document}
\maketitle

\begin{center}
\begin{tabular}{|p{100pt}|p{100pt}|p{100pt}|p{100pt}|}
\hline
题目名称 & 阶乘 & 激光和镜子 & 干草堆猜测\\
\hline
目录 & \verb|fact| & \verb|lasers| & \verb|bales|\\
\hline
可执行文件名 & \verb|fact| & \verb|lasers| & \verb|bales|\\
\hline
输入文件名 & \verb|fact.in| & \verb|lasers.in| & \verb|bales.in|\\
\hline
输出文件名 & \verb|fact.out| & \verb|lasers.out| & \verb|bales.out|\\
\hline
时间限制 & 1s & 1s & 1s\\
\hline
空间限制 & 512MB & 512MB & 512MB\\
\hline
测试点数量 & 20 & 20 & 20\\
\hline
测试点分数 & 5 & 5 & 5\\
\hline
比较方式 & SPJ & 全文 & 全文\\
\hline
部分分 & 有 & 无 & 无\\
\hline
\end{tabular}
\end{center}
提交源程序文件名
\begin{center}
\begin{tabular}{|p{60pt}p{28pt}|p{100pt}|p{100pt}|p{100pt}|}
\hline
对于C++ & 语言 & \verb|fact.cpp|& \verb|lasers.cpp| & \verb|bales.cpp| \\
\hline
对于C & 语言 & \verb|fact.c| & \verb|lasers.c| & \verb|bales.c|\\
\hline
对于Pascal & 语言 & \verb|fact.pas|& \verb|lasers.pas| & \verb|bales.pas|\\
\hline
\end{tabular}
\end{center}
编译选项
\begin{center}
\begin{tabular}{|p{60pt}p{28pt}|p{100pt}|p{100pt}|p{100pt}|}
\hline
对于C++ & 语言 & \verb|-O2 -std=gnu++11| & \verb|-O2 -std=gnu++11| & \verb|-O2 -std=gnu++11|\\
\hline
对于C & 语言 & \verb|-O2| & \verb|-O2| & \verb|-O2|\\
\hline
对于Pascal & 语言 & \verb|-O2| & \verb|-O2| & \verb|-O2|\\
\hline
\end{tabular}
\end{center}
注意事项:
\begin{enumerate}
\item 注意编译选项,避免未定义行为或编译错误。
\item 代码长度限制为100KB。
\item 注意代码常数和I/O造成的效率影响。
\end{enumerate}

\newpage

\section{阶乘(fact.cpp/c/pas)}

\subsection{题目描述}

定义$n!=\prod_{i=1}^n n\;\;\forall n\ge1$。请求出$n!$($1\le n\le1,000,000,000$)的近似值,保留$k$位($k\le10$)有效数字。

\subsection{输入格式(fact.in)}

两个整数$n,k$。

\subsection{输出格式(fact.out)}

一个用科学记数法表示的答案,格式为\verb|d.ddddde+dddd|,其中d表示数字。当然,实际的长度与$n$和$k$有关。\underline{\textbf{不要输出末尾的0,如果没有小数部分,不要输出小数点。}}

\subsection{输入样例}

\begin{verbatim}
10 4
\end{verbatim}

\subsection{输出样例}

\begin{verbatim}
3.629e+6
\end{verbatim}

\subsection{样例解释}

$10!=3,628,800\approx3.629*10^6$

\subsection{数据范围}

\begin{center}
\begin{tabular}{P{50pt}|P{180pt}|P{180pt}}
\Xhline{3\arrayrulewidth}
测试点 & $n$ & $k$\\
\Xhline{2\arrayrulewidth}
1 & $\le20$ & \multirow{13}{*}{$\le6$}\\
\cline{1-2}
2 & $\le100$\\
\cline{1-2}
3 & $\le150$\\
\cline{1-2}
4 & $\le500$\\
\cline{1-2}
5 & $\le1,000$\\
\cline{1-2}
6 & $\le1,500$\\
\cline{1-2}
7 & $\le2,000$\\
\cline{1-2}
8 & $\le3,000$\\
\cline{1-2}
9 & $\le5,000$\\
\cline{1-2}
10 & $\le10,000$\\
\cline{1-2}
11 & $\le50,000$\\
\cline{1-2}
12 & $\le3,000,000$\\
\cline{1-2}
13 & $\le10,000,000$\\
\hline
14 & \multirow{2}{*}{$\le1,000,000$} & $\le7$\\
\cline{1-1}\cline{3-3}
15 && $\le8$\\
\hline
16 & \multirow{2}{*}{$\le100,000,000$} & $\le9$\\
\cline{1-1}\cline{3-3}
17 && \multirow{4}{*}{$\le10$}\\
\cline{1-2}
18 & $\le200,000,000$\\
\cline{1-2}
19 & $\le500,000,000$\\
\cline{1-2}
20 & $\le1,000,000,000$\\
\Xhline{3\arrayrulewidth}
\end{tabular}
\end{center}

\subsection{部分分}

\begin{itemize}
\item 如果你的答案格式错误,不得分
\item 如果你的答案格式正确,并且e前的部分完全正确,得到测试点60\%的分数
\item 如果你的答案格式正确,并且e后的部分完全正确,得到测试点40\%的分数
\end{itemize}

\newpage

\section{激光和镜子(lasers.cpp/c/pas)}

\subsection{题目描述}

因为一些原因,农夫约翰的奶牛总是喜欢进行激光展示。

为了它们最新的展示,奶牛们已经取得了一个巨大而强大的激光源------它是如此的巨大,实际上,以至于它们看起来不能轻松的把它从交付的地方移动。它们想找到一种方法把激光从光源处发送到在另一边的牛棚。光源和牛棚可以看成在二维平面上的点。奶牛们打算旋转光源使其沿水平或竖直方向发出一束光(也就是和x轴或y轴平行)。它们将会通过一些镜子反射激光,使其重定向到牛棚。

在农场上有$N$个\underline{\textbf{不同}}的栅栏柱($1\le N\le300,000$)在二维平面上(\underline{\textbf{不同于光源和}}\\\underline{\textbf{牛棚}}),奶牛可以把镜子挂载在栅栏柱上面。奶牛也可以选择不在一个栅栏柱上挂载镜子,这种情况下激光将会简单地直接从上面通过而\underline{\textbf{不改变方向}}。如果奶牛在栅栏柱上挂载了镜子,它们可以把镜子对齐像/或\textbackslash,这样它将会把一束水平的光重定向为竖直方向,反之亦然。

请计算奶牛\underline{\textbf{最少}}需要多少镜子来把激光重定向到牛棚,\underline{\textbf{数据保证有解}}。

\subsection{输入格式(lasers.in)}

第一行包含五个整数$N,x_L,y_L,x_B,y_B$,其中$(x_L,y_L)$是光源所在的位置,$(x_B,y_B)$是牛棚所在的位置。所有的坐标都在0到1,000,000,000之间。

接下来$N$行,每行包含一个栅栏柱的坐标,也在$0\dots1,000,000,000$范围内。

\subsection{输出格式(lasers.out)}

输出最少需要的镜子的数量。

\subsection{输入样例}

\begin{verbatim}
4 0 0 7 2
3 2
0 2
1 6
3 0
\end{verbatim}

\subsection{输出样例}

\begin{verbatim}
1
\end{verbatim}

\subsection{样例解释}

沿竖直方向发出一束光,在(0,2)处放置/的镜子,即可重定向到牛棚。

\subsection{数据范围}

\begin{center}
\begin{tabular}{P{50pt}|P{180pt}|P{180pt}}
\Xhline{3\arrayrulewidth}
测试点 & $N$ & 坐标范围\\
\Xhline{2\arrayrulewidth}
1 & $\le5$ & \multirow{2}{*}{$0\dots100$}\\
\cline{1-2}
2 & $\le10$\\
\hline
3 & $\le50$ & \multirow{4}{*}{$0\dots200$}\\
\cline{1-2}
4 & $\le100$\\
\cline{1-2}
5 & $\le200$\\
\cline{1-2}
6 & $\le500$\\
\hline
7 & $\le1,000$ & \multirow{4}{*}{$0\dots2,000$}\\
\cline{1-2}
8 & $\le2,000$\\
\cline{1-2}
9 & $\le5,000$\\
\cline{1-2}
10 & \multirow{2}{*}{$\le10,000$}\\
\cline{1-1}\cline{3-3}
11 && $0\dots1,000,000,000$\\
\hline
12 & \multirow{3}{*}{$\le20,000$} & $0\dots2,000$\\
\cline{1-1}\cline{3-3}
13 && $0\dots100,000$\\
\cline{1-1}\cline{3-3}
14 && $0\dots1,000,000,000$\\
\hline
15 & \multirow{2}{*}{$\le50,000$} & $0\dots100,000$\\
\cline{1-1}\cline{3-3}
16 && $0\dots1,000,000,000$\\
\hline
17 & \multirow{2}{*}{$\le100,000$} & $0\dots100,000$\\
\cline{1-1}\cline{3-3}
18 && $0\dots1,000,000,000$\\
\hline
19 & \multirow{2}{*}{$\le300,000$} & $0\dots150,000$\\
\cline{1-1}\cline{3-3}
20 && $0\dots1,000,000,000$\\
\Xhline{3\arrayrulewidth}
\end{tabular}
\end{center}

\newpage

\section{干草堆猜测(bales.cpp/c/pas)}

\subsection{题目描述}

奶牛们设计了一个猜数游戏,来锻炼它们的逻辑推理能力。

游戏开始前,一头奶牛会在牛棚后面摆$N$堆干草($1\le N\le1,000,000$);每堆干草有若干捆,数量在$1\dots1,000,000,000$之间,\underline{\textbf{并且没有两堆中的草一样多}}。所有干草堆排成一条直线,从左到右编号$1\dots N$。游戏开始后,参与游戏的奶牛会问$Q$个问题($1\le Q\le300,000$):编号为$Q_l\dots Q_r$的草堆中($1\le Q_l\le Q_r\le N$),最小的那堆里有多少捆草?

对于每个问题,摆干草的奶牛都要给出回答$A$。奶牛们当然会做这个RMQ问题了,但是现在,它们不知道每堆干草的数量。请你计算摆干草的奶牛\underline{\textbf{第一个}}自相矛盾的回答。

\subsection{输入格式(bales.in)}

第一行两个整数$N,Q$。

接下来$Q$行,每行三个整数$Q_l,Q_r,A$,描述了一个问题以及其回答。

\subsection{输出格式(bales.out)}

如果摆干草的奶牛\underline{\textbf{有可能完全正确地回答了所有问题}},也就是说,能找到一种使得所有回答都合理的摆放干草的方法,\underline{\textbf{输出0}}。否则输出一个$1\dots Q$之间的数,表示第一个自相矛盾的回答的编号。

\subsection{输入样例}

\begin{verbatim}
20 4
1 10 7
5 19 7
3 12 8
11 15 12
\end{verbatim}

\subsection{输出样例}
\begin{verbatim}
3
\end{verbatim}

\subsection{样例解释}

第三个问题的回答与前两个回答矛盾。因为每堆中的草的数量唯一,从前两个回答中我们能推断出,编号$5\dots10$的干草堆中最小的那堆有7捆干草。很显然,第三个问题的回答与这个推断矛盾。

\subsection{数据范围}

\begin{center}
\begin{tabular}{P{40pt}|P{100pt}|P{100pt}|P{100pt}|P{40pt}}
\Xhline{3\arrayrulewidth}
测试点 & $N$ & $Q$ & $A$ & 性质 \\
\Xhline{2\arrayrulewidth}
1 & $\le5$ & \multirow{3}{*}{$\le10$} & $\le5$ & \multirow{2}{*}{=1} \\
\cline{1-2}\cline{4-4}
2 & $\le20$ && $\le20$ \\
\cline{1-2}\cline{4-5}
3 & $\le100$ && $\le500$ & =2 \\
\hline
4 & $\le10$ & \multirow{2}{*}{$\le20$} & $\le10$ & =1 \\
\cline{1-2}\cline{4-5}
5 & $\le10,000$ && $\le50,000$ & =2 \\
\hline
6 & $\le1,000$ & $\le50$ & $\le5,000$ & =1 \\
\hline
7 & $\le100,000$ & $\le100$ & $\le500,000$ & =2 \\
\hline
8 & $\le10,000$ & $\le200$ & $\le50,000$ & =1 \\
\hline
9 & $\le1,000,000$ & $\le500$ & $\le5,000,000$ & \multirow{2}{*}{=2}\\
\cline{1-4}
10 & \multirow{4}{*}{$\le100,000$} & $\le1,000$ & $\le500,000$\\
\cline{1-1}\cline{3-5}
11 && $\le2,000$ & \multirow{2}{*}{$\le1,000,000,000$} & =1\\
\cline{1-1}\cline{3-3}\cline{5-5}
12 && $\le5,000$ && =2\\
\cline{1-1}\cline{3-5}
13 && \multirow{2}{*}{$\le20,000$} & $\le100,000$ & \multirow{3}{*}{=1} \\
\cline{1-2}\cline{4-4}
14 & $\le500,000$ && $\le500,000$\\
\cline{1-4}
15 & $\le100,000$ & $\le50,000$ & \multirow{6}{*}{$\le1,000,000,000$}\\
\cline{1-3}\cline{5-5}
16 & \multirow{2}{*}{$\le1,000,000$} & \multirow{2}{*}{$\le100,000$} && =2\\
\cline{1-1}\cline{5-5}
17 &&&& =3\\
\cline{1-3}\cline{5-5}
18 & $\le300,000$ & \multirow{3}{*}{$\le300,000$} && \multirow{3}{*}{=1}\\
\cline{1-2}
19 & \multirow{2}{*}{$\le1,000,000$} &&&\\
\cline{1-1}
20 &&&& \\
\Xhline{3\arrayrulewidth}
\end{tabular}
\end{center}

\paragraph{性质}
\begin{enumerate}
\item 无特殊性质
\item 所有的$A$互不相同
\item 所有的$A$相同
\end{enumerate}

\end{document}
