\documentclass[hyperref,UTF8,12pt,a4paper]{ctexart}

\usepackage{amsmath}

\usepackage{geometry}
\geometry{left=1in,right=1in,top=1in,bottom=1in}

\usepackage{titling}
\pretitle{\begin{center}\fontsize{30pt}{30pt}\selectfont}
\posttitle{\end{center}}

\title{NOIp Senior Day2 Solution}
\author{zhzh2001}
\date{}

\begin{document}
\maketitle

\section{围栏}

\subsection{思路}

一个正方形可以由左上角坐标和边长确定,只要枚举所有正方形就能找到答案。这样的时间复杂度=枚举左上角坐标的时间*枚举边长的时间*统计正方形内的草场的时间。直接枚举+离散化是$O(N^4)$的。我们可以做二维前缀和,并二分枚举边长,并用单调性在$O(N^2)$的时间内判断,总的复杂度是$O(N^2\log N)$的。

\subsection{常数问题}

由于$N\le3,000$,坐标范围$0\dots10^{18}$,上述方法较难直接通过。而官方的实现常数小,也利用了单调性,但是时间复杂度难以分析,平均情况下比二分慢,但是在较坏情况下比二分快,而且与坐标范围无关。

\subsection{总结}

本题来自\verb|USACO06JAN Gold T3:Corral the Cows|

主要考察了二分和单调性。

\section{Quicksort Killer}

\subsection{直接排序}

在我们印象中,快速排序在正序或逆序时会达到$\Theta(N^2)$。因此直接对$N$个数排序输出。这样就有50分了,额外的10分来自于随机种子等于0的情况,此时和第二种实现完全一样。

\subsection{思路}

要使快速排序达到最坏情况,只要让基准值是当前序列中最小的即可(最大也可以)。分析可以发现,这样只会扫描一遍,并把基准值与第一个数交换,$T(N)=T(N-1)+\Theta(N)=\Theta(N^2)$。具体实现维护一个数组B记录原先的位置,每次只要在基准值填上当前最小数,并交换$B[l]$和$B[F[l, r]]$。

\subsection{时间计算}

这也是比较容易出错的部分,可以直接枚举,也可以推出公式\footnote{参见\url{http://pubs.opengroup.org/onlinepubs/9699919799/basedefs/V1_chap04.html\#tag_04_16}}:

\begin{verbatim}
tm_sec + tm_min*60 + tm_hour*3600 + tm_yday*86400 +
    (tm_year-70)*31536000 + ((tm_year-69)/4)*86400 -
    ((tm_year-1)/100)*86400 + ((tm_year+299)/400)*86400
\end{verbatim}

其中\verb|tm_yday|表示从1月1日到该时刻经过的天数。

另一个问题是在Windows下,计算结果必须减掉8*3600秒,因为我们的时区为UTC+8。这也是我强调UTC的原因。

\subsection{使用日期函数计算}

实际上,C语言的\verb|time.h|中有一个\verb|mktime|函数可以直接完成转换。其原型如下:

\begin{verbatim}
time_t mktime( struct tm *time );
\end{verbatim}

而上述公式中的值都是结构体\verb|tm|中的成员,另外还有\verb|tm_mon|和\verb|tm_mday|,可以替换\verb|tm_yday|。但是请注意这些值的范围。\footnote{参见\url{http://en.cppreference.com/w/c/chrono/tm}}

\subsection{总结}

本题是我的原创题,考察了对快速排序的理解,以及日期模拟。

\section{花园改造}

\subsection{暴力}

可以把当前$N$个花坛内的泥土状压或hash,然后就转化为最多$C^N$个点的最短路。(设$A_i,B_i\le C$)

\subsection{动态规划}

通过一个巧妙的模型转化,就可以把这题转化为一个字符串编辑距离问题。只要按顺序把每个花坛$i$内的泥土数量展开成对应数量的$i$,如1,2,3,4转化为1,2,2,3,3,3,4,4,4,4。接下来,购买操作就可以转化为插入一个对应的数字(花费$X$),移除操作转化为删除一个对应的数字(花费$Y$),运送操作转化为修改一个对应的数字(花费$Z*|i-j|$)。这样时间复杂度$O((NC)^2)$。

\subsection{贪心}

官方题解时间复杂度为$O(NC)$,但是分析很复杂,有兴趣可以去研究。

这里介绍一种CF上有人提出的贪心做法。如果$A_i>B_i$,对于每个单位多余的泥土,可以花费$Y$把它移除,或者运送到前面的花坛$j$花费$Z*(i-j)$(这里强制$i>j$,$i<j$的情况会在$A_i<B_i$时考虑)。我们如何知道哪种更优呢?$Z*(i-j)=Z*i-Z*j$,只要找到最小的$-Z*j$就可以了。当然我们还要考虑退钱,因为可能在一开始直接移除更优,而后来发现运送更优,这时就要退钱。

一般的,我们维护两个小根堆,分别维护多余和缺少的泥土的$-Z*i-cost$,其中$cost$为原费用,初始时为$Y$和$X$。如果$A_i>B_i$,$cost=min(Y,Q_{need}.top+Z*i)$,然后向多余堆插入$-Z*i-cost$;$A_i<B_i$同理。时间复杂度$O(NC\log NC)$,空间复杂度$O(NC)$。

\end{document}