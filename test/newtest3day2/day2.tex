\documentclass[hyperref,UTF8,12pt,a4paper]{ctexart}

\usepackage{amsmath,multirow,makecell,algorithm}

\usepackage{array}
\newcolumntype{P}[1]{>{\centering\arraybackslash}p{#1}}

\usepackage{geometry}
\geometry{left=1in,right=1in,top=1in,bottom=1in}

\usepackage{titling}
\pretitle{\begin{center}\fontsize{30pt}{30pt}\selectfont}
\posttitle{\end{center}}

\usepackage{algpseudocode}

\title{NOIp Senior Day2}
\author{zhzh2001}
\date{}

\begin{document}
\maketitle

\begin{center}
\begin{tabular}{|p{100pt}|p{100pt}|p{100pt}|p{100pt}|}
\hline
题目名称 & 围栏 & Quicksort Killer & 花园改造\\
\hline
目录 & \verb|corral| & \verb|qkiller| & \verb|landscape|\\
\hline
可执行文件名 & \verb|corral| & \verb|qkiller| & \verb|landscape|\\
\hline
输入文件名 & \verb|corral.in| & \verb|qkiller.in| & \verb|landscape.in|\\
\hline
输出文件名 & \verb|corral.out| & \verb|qkiller.out| & \verb|landscape.out|\\
\hline
时间限制 & 1s & 1s & 1s\\
\hline
空间限制 & 512MB & 512MB & 512MB\\
\hline
测试点数量 & 20 & 10 & 20\\
\hline
测试点分数 & 5 & 10 & 5\\
\hline
比较方式 & 全文 & SPJ & 全文\\
\hline
部分分 & 无 & 无 & 无\\
\hline
\end{tabular}
\end{center}
提交源程序文件名
\begin{center}
\begin{tabular}{|p{60pt}p{28pt}|p{100pt}|p{100pt}|p{100pt}|}
\hline
对于C++ & 语言 & \verb|corral.cpp|& \verb|qkiller.cpp| & \verb|landscape.cpp| \\
\hline
对于C & 语言 & \verb|corral.c| & \verb|qkiller.c| & \verb|landscape.c|\\
\hline
对于Pascal & 语言 & \verb|corral.pas|& \verb|qkiller.pas| & \verb|landscape.pas|\\
\hline
\end{tabular}
\end{center}
编译选项
\begin{center}
\begin{tabular}{|p{60pt}p{28pt}|p{100pt}|p{100pt}|p{100pt}|}
\hline
对于C++ & 语言 & \verb|-O2 -std=gnu++11| & \verb|-O2 -std=gnu++11| & \verb|-O2 -std=gnu++11|\\
\hline
对于C & 语言 & \verb|-O2| & \verb|-O2| & \verb|-O2|\\
\hline
对于Pascal & 语言 & \verb|-O2| & \verb|-O2| & \verb|-O2|\\
\hline
\end{tabular}
\end{center}
注意事项:
\begin{enumerate}
\item 注意编译选项,避免未定义行为或编译错误。
\item 代码长度限制为100KB。
\item 注意代码常数和I/O造成的效率影响。
\end{enumerate}

\newpage

\section{围栏(corral.c/cpp/pas)}
% sorrel...

\subsection{题目描述}

农夫约翰打算建造一个围栏来圈养他的奶牛。作为最挑剔的兽类,奶牛们要求这个围栏\underline{\textbf{必须是正方形}}的,并且围栏里\underline{\textbf{至少}}要有$C$个草场,来供应它们的食物。

约翰的土地上一共有$N$个草场($1\le C\le N\le3,000$),每个草场都在一块\underline{\textbf{1x1的方}}\\\underline{\textbf{格}}内,坐标范围$0\dots10^{18}$。如果有多个草场在同一个方格内,那么\underline{\textbf{它们的坐标就会相同}}。

请帮助约翰计算围栏的最小边长。

\subsection{输入格式(corral.in)}

第一行两个整数$C,N$。

接下来$N$行,每行两个整数,表示一个草场所在方格的坐标。

\subsection{输出格式(corral.out)}

输出最小边长。

\subsection{输入样例}

\begin{verbatim}
3 4
1 2
2 1
4 1
5 2
\end{verbatim}

\subsection{输出样例}

\begin{verbatim}
4
\end{verbatim}

\subsection{样例解释}

围栏左上角坐标可以为(1,1),边长为4,此时前三个草场都在围栏内。更小的边长不能满足要求。

\subsection{数据范围}

\begin{center}
\begin{tabular}{P{50pt}|P{180pt}|P{180pt}}
\Xhline{3\arrayrulewidth}
测试点 & $N$ & 坐标范围\\
\Xhline{2\arrayrulewidth}
1 & \multirow{2}{*}{$\le5$} & $0\dots10$\\
\cline{1-1}\cline{3-3}
2 && $0\dots10^{18}$\\
\hline
3 & $\le10$ & $0\dots1,000$\\
\hline
4 & \multirow{3}{*}{$\le50$} & $0\dots50$\\
\cline{1-1}\cline{3-3}
5 && $0\dots200$\\
\cline{1-1}\cline{3-3}
6 && $0\dots10^{18}$\\
\hline
7 & \multirow{3}{*}{$\le200$} & $0\dots200$\\
\cline{1-1}\cline{3-3}
8 && $0\dots5,000$\\
\cline{1-1}\cline{3-3}
9 && $0\dots10^{18}$\\
\hline
10 & \multirow{3}{*}{$\le500$} & $0\dots500$\\
\cline{1-1}\cline{3-3}
11 && $0\dots5,000$\\
\cline{1-1}\cline{3-3}
12 && $0\dots10^{18}$\\
\hline
13 & \multirow{2}{*}{$\le1,000$} & $0\dots5,000$\\
\cline{1-1}\cline{3-3}
14 && $0\dots10^{18}$\\
\hline
15 & \multirow{2}{*}{$\le2,000$} & $0\dots500$\\
\cline{1-1}\cline{3-3}
16 && $0\dots10^{18}$\\
\hline
17 & \multirow{4}{*}{$\le3,000$} & $0\dots200$\\
\cline{1-1}\cline{3-3}
18 && \multirow{2}{*}{$0\dots1,000,000,000$}\\
\cline{1-1}
19 &&\\
\cline{1-1}\cline{3-3}
20 && $0\dots10^{18}$\\
\Xhline{3\arrayrulewidth}
\end{tabular}
\end{center}

\newpage

\section{Quicksort Killer(qkiller.c/cpp/pas)}

\subsection{题目描述}

快速排序是一种广为使用的排序算法,其最好和平均时间复杂度均为$\Theta(N\log N)$,但是最坏时间复杂度可达$\Theta(N^2)$。为了简化问题,我们考虑给$N$个32位\underline{\textbf{随机}}整数排序($N=100,000$)。

你的任务是,给定这$N$个整数,\underline{\textbf{重新排列}}这些整数,使得特定的快速排序实现运行时间超过1s。

我们考虑一种常见的实现,来自\verb|Free Pascal|的示例代码:

\begin{algorithm}
\begin{algorithmic}[1]
\Procedure{Sort}{$A,l,r$}\Comment{对$A[l\dots r]$排序}
\State{$i\gets l$}
\State{$j\gets r$}
\State{$x\gets A$[\Call{F}{$l,r$}]}\Comment{$x$为\emph{基准值}}
\Repeat
\While{$A[i]<x$}
\State{$i\gets i+1$}
\EndWhile
\While{$x<A[j]$}
\State{$j\gets j-1$}
\EndWhile
\If{$i\le j$}
\State{\Call{Swap}{$A[i],A[j]$}}
\State{$i\gets i+1$}
\State{$j\gets j-1$}
\EndIf
\Until{$i>j$}
\If{$l<j$}
\State{\Call{Sort}{$l,j$}}
\EndIf
\If{$i<r$}
\State{\Call{Sort}{$i,r$}}
\EndIf
\EndProcedure
\end{algorithmic}
\end{algorithm}

其中\verb|F(l, r)|返回一个[l, r]之间的数,作为\emph{基准值}的下标。这里,你需要考虑四种实现:

\begin{enumerate}
\item F(l, r) = (l + r) / 2
\item F(l, r) = l
\item F(l, r) = r
\item F(l, r) = random(l, r)
\end{enumerate}

上述除法为整数除法,\verb|random(l, r)|函数返回[l, r]间的伪随机数。随机函数的工作原理如下:

在程序开始时,自动初始化随机数种子$x_0 = time(NULL)$,\verb|time(NULL)|函数同C/C++中同名函数,即返回从\underline{\textbf{UTC 1970/01/01 00:00:00}}开始到调用时经过的\underline{\textbf{秒数}}。(不必考虑闰秒等问题)

\verb|random(l, r)|函数第$i$次调用时$x_i = x_{i-1} * 48,271 \mod 2,147,483,647$(即\\\verb|minstd_rand|),返回值为$x_i \mod (r-l+1) +l$。

\subsection{输入格式(qkiller.in)}

第一行一个整数$type$($type\in[1,4]$),表示F函数的实现。

如果$type=4$,接下来一行一个时间,格式为\verb|yyyy/mm/dd hh:mm:ss|,表示调用时间,范围从UTC 1970/1/1 00:00:00到2020/12/31 23:59:59。

接下来一行$N$个整数,在32位有符号整数范围内。

\subsection{输出格式(qkiller.out)}

输出$N$个符合要求的整数。

\subsection{数据范围}

\begin{center}
\begin{tabular}{P{50pt}|P{180pt}|P{180pt}}
\Xhline{3\arrayrulewidth}
测试点 & $type$ & 整数范围\\
\Xhline{2\arrayrulewidth}
1 & \multirow{2}{*}{=1} & $[-2^{31},2^{31})$\\
\cline{1-1}\cline{3-3}
2 && $[-10^8,10^8]$\\
\hline
3 & \multirow{2}{*}{=2} & $[-2^{31},2^{31})$\\
\cline{1-1}\cline{3-3}
4 && $[-1,000,1,000]$\\
\hline
5 & \multirow{2}{*}{=3} & $[-2^{31},2^{31})$\\
\cline{1-1}\cline{3-3}
6 && $[-1,000,1,000]$\\
\hline
7 & \multirow{4}{*}{=4} & $[-2^{31},2^{31})$\\
\cline{1-1}\cline{3-3}
8 && \multirow{3}{*}{$[-10^8,10^8]$}\\
\cline{1-1}
9 &&\\
\cline{1-1}
10 &&\\
\Xhline{3\arrayrulewidth}
\end{tabular}
\end{center}

对于100\%的数据,整数都在对应的范围内\underline{\textbf{等概率随机}}生成。

\newpage

\section{花园改造(landscape.c/cpp/pas)}

\subsection{题目描述}

农夫约翰正在建造一个漂亮的花园,并且在这个工程中需要移动大量的泥土。

这个花园可以看作一个由$N$个花坛组成的序列($1\le N\le100,000$),花坛$i$最初包含$A_i$个单位的泥土。农夫约翰想要改造这个花园,使得花坛$i$包含$B_i$个单位的泥土。$A_i,B_i$都是$0\dots50$之间的整数。

为了改造花园,农夫约翰有几种选项:

\begin{itemize}
\item 花费$X$元购买一个单位泥土,并把它放在一个他选择的花坛中。
\item 花费$Y$元移除他选择的花坛中一个单位泥土,并把它运走。
\item 花费$Z*|i-j|$元把一个单位泥土从花坛$i$运送到花坛$j$。
\end{itemize}

请计算完成改造计划的最小花费。

\subsection{输入格式(landscape.in)}

第一行四个整数$N,X,Y,Z$($0\le X,Y\le10^8;0\le Z\le1,000$)。

接下来$N$行,每行两个整数$A_i,B_i$。

\subsection{输出格式(landscape.out)}

输出完成改造计划的最小花费。

\subsection{输入样例}

\begin{verbatim}
4 100 200 1
1 4
2 3
3 2
4 0
\end{verbatim}

\subsection{输出样例}

\begin{verbatim}
210
\end{verbatim}

\subsection{样例解释}

有一个单位的泥土必须被移除(从第四个花坛),花费200元。剩下的泥土可以花费10元移动(三个单位泥土从第四个花坛到第一个,一个单位从第三个到第二个)。

\subsection{数据范围}

\begin{center}
\begin{tabular}{P{50pt}|P{180pt}|P{180pt}}
\Xhline{3\arrayrulewidth}
测试点 & $N$ & $A_i,B_i$\\
\Xhline{2\arrayrulewidth}
1 & $\le3$ & $\le3$ \\
\hline
2 & $\le4$ & $\le4$ \\
\hline
3 & $\le5$ & \multirow{2}{*}{$\le5$} \\
\cline{1-2}
4 & \multirow{2}{*}{$\le10$} & \\
\cline{1-1}\cline{3-3}
5 & & \multirow{5}{*}{$\le10$} \\
\cline{1-2}
6 & $\le20$ & \\
\cline{1-2}
7 & $\le30$ & \\
\cline{1-2}
8 & $\le50$ & \\
\cline{1-2}
9 & \multirow{4}{*}{$\le100$} & \\
\cline{1-1}\cline{3-3}
10 & & $\le20$ \\
\cline{1-1}\cline{3-3}
11 & & $\le30$ \\
\cline{1-1}\cline{3-3}
12 & & \multirow{9}{*}{$\le50$} \\
\cline{1-2}
13 & $\le500$ & \\
\cline{1-2}
14 & $\le1,000$ & \\
\cline{1-2}
15 & $\le2,000$ & \\
\cline{1-2}
16 & $\le5,000$ & \\
\cline{1-2}
17 & $\le20,000$ & \\
\cline{1-2}
18 & $\le50,000$ & \\
\cline{1-2}
19 & \multirow{2}{*}{$\le100,000$} & \\
\cline{1-1}
20 & & \\
\Xhline{3\arrayrulewidth}
\end{tabular}
\end{center}

\end{document}